\documentclass[a4paper,draft]{amsproc}
\usepackage{amssymb}
\usepackage{amscd} %% Package for commutative diagrams

\theoremstyle{plain}
 \newtheorem{thm}{Theorem}[section]
 \newtheorem{prop}{Proposition}[section]
 \newtheorem{lem}{Lemma}[section]
 \newtheorem{cor}{Corollary}[section]
\theoremstyle{definition}
 \newtheorem{exm}{Example}[section]
 \newtheorem{dfn}{Definition}[section]
\theoremstyle{remark}
 \newtheorem{rem}{Remark}[section]
 \numberwithin{equation}{section}

%% Please, do not change the following four lines:
\renewcommand{\le}{\leqslant}\renewcommand{\leq}{\leqslant}
\renewcommand{\ge}{\geqslant}\renewcommand{\geq}{\geqslant}
\renewcommand{\setminus}{\smallsetminus}
\setlength{\textwidth}{28cc} \setlength{\textheight}{42cc}

\title[]{Elementary Facts About Primes}

\author[Miliyon T.]{\bfseries Miliyon T.}

\address{
Department of Mathematics \\ % \hfill (Received 00 00 2010)\\
Addis Ababa University   \\ %\hfill (Revised  00 00 2010)\\
Addis Ababa\\
Ethiopia}
\email{miliyon@ymail.com}


\thanks{Structure in randomness of primes} %% optional

\begin{document}

\vspace{18mm} \setcounter{page}{1} \thispagestyle{empty}


\begin{abstract}
Prime numbers are very important figures in number theory. They are like atom that build up a molecule by building up the integers because of the fact fundamental theorem of arithmetic. For centuries Mathematicians tried to understand prime numbers. Despite their hard work too little is known about primes. In this paper we will try to look at some of the results that have been discovered so far.
\end{abstract}

\maketitle

\section{Histroy}  %% Please avoid complicated formulas in titles

The earliest surviving records of the explicit study of prime numbers come from the Ancient Greeks.  Euclid's Elements
(circa 300 BC) contain important theorems about primes, including the infinitude of primes  and the fundamental theorem of arithmetic. Euclid also showed how to construct a perfect number  from a  Mersenne prime. The Sieve of Eratosthenes,
attributed to Eratosthenes, is a simple method to compute primes.

After the Greeks, little happened with the study of prime numbers until the 17th century. In 1640 Pierre de Fermat
stated Fermat's little theorem. It was proved later by Euler. Euler ever gave its generalization. Fermat also conjectured that all numbers of the form $2^{2^n} + 1$ are prime and he verified this up to $n = 4$. However, the very next Fermat number $2^{2^5} + 1$ is composite (one of its prime factors is 641), as Euler discovered later(Feramt conjectures Euler proves it!). Numbers of the form $2^{2^n} + 1$ are now called Fermat numbers. The French monk  Marin Mersenne looked at primes of the form $2^p -1$, with p a prime. They are called Mersenne primes in his honor.

Euler's work in number theory included many results about primes. Some of his major results are the divergence of the sum of the reciprocal of primes, Euler product and  in 1747 he showed that the even  perfect numbers are precisely the integers of the form $2^{p -1}(2^p -1)$, where the second factor is a Mersenne prime.

At the start of the 19th century, Legendre and Gauss independently conjectured the prime number theorem. Ideas of Riemann in his  1859 paper on the zeta-function sketched a program that would lead to a proof of the prime number theorem. This outline was completed by  Hadamard  and  de la Vallée Poussin, who independently proved the  prime number theorem  in 1896.

Proving a number is prime is not done (for large numbers) by trial division. Many mathematicians have worked on
primality tests for large numbers, often restricted to specific number forms. This includes  Pépin's test  for Fermat
numbers (1877),  Proth's theorem  (around 1878), the  Lucas–Lehmer primality test  (originated 1856), and the
generalized Lucas primality test. More recent algorithms like APRT-CL, ECPP, and AKS work on arbitrary numbers
but remain much slower.

For a long time, prime numbers were thought to have extremely limited application outside of pure mathematics. This changed in the 1970s when the concepts of  public-key cryptography  were invented, in which prime numbers formed the basis of the first algorithms such as the RSA cryptosystem algorithm.

\section{Definition and Facts}

\begin{dfn}
A \textbf{prime number} is a natural number greater than 1 that has no positive divisors other than 1 and itself.
\end{dfn}

\begin{dfn}
If a natural number greater than 1 is not a prime, then its called \textbf{composite}.
\end{dfn}
\subsection{Fundamental Theorem of Arithmetic}
\begin{thm}\label{fundamental}
Every positive integer greater than one is either prime or can be \textbf{uniquely} factored in to \textit{prime} numbers.
\end{thm}

\begin{proof}
Either n is a prime or it is composite. In the first case there is nothing to prove. If n is composite, then there exists a prime divisor of n, as we have shown. Thus, n may be written as $n=p_1n_1,$ where $p_1$ is prime and $1<n_1<n.$ If $n_1$ is prime, then we have our representation. In the contrary case, the argument is repeated to produce a second prime number $p_2$ such that $n_1=p_2n_2$; that is,
$$
n=p_1p_2n_2; 1<n_2<n_1:
$$

 If $n_2$ is a prime, then it is not necessary to go further. Otherwise, write\\ $n_2=p_3n_3,$ with $p_3$ a prime; hence,
$$
N=p_1\cdot p_2\cdot p_3\cdot n_3; 1<n_3<n_2:
$$

The decreasing sequence
$n>n_1>n_2>\cdot\cdot\cdot >1$
Cannot continue indefinitely, so that after a finite number of steps $n_k$ is a prime, say $p_k$. This leads to the prime factorization
$$n=p_1 p_2\cdot\cdot\cdot p_k:$$
The second part of the proof the uniqueness of the prime factorization is more difficult. To this purpose let us suppose that the integer n can be represented as a product of primes in two ways; say,
$$n=p_1 p_2\cdot\cdot\cdot p_r=q_1q_2\cdot\cdot\cdot q_s; r\leq s;$$
Where the $p_i$ and $q_j$ are all primes, written in increasing order, so that
$ p_1\leq p_2 \leq\cdot\cdot\cdot\leq p_r$ and $q_1\leq q_2 \leq\cdot\cdot\cdot\leq q_s:$
Because $p_1\mid q_1q_2\cdot\cdot\cdot q_s,$ we know that $p_1|q_k$ for some value of k. Being a prime, $q_k$ has only two divisors, 1 and itself. Because $p_1$ is greater than 1, we must conclude that $p_1=q_k;$ but then it must be that $p_1\geq q_1.$ An entirely similar argument (starting with $q_1$ rather than $p_1$) yields $q_1\geq p_1,$ so that in fact $p_1=q_1.$ We can cancel this common factor and obtain
$$
p_2 p_3\cdot\cdot\cdot p_r=q_2q_3\cdot\cdot\cdot q_s:
$$
Now repeat the process to get $p_2=q_2;$ cancel again, to see that
$$
p_3 p_4\cdot\cdot\cdot p_r=q_3q_4\cdot\cdot\cdot q_s:
$$
Continue in this fashion. If the inequality $r <s$ held, we should eventually arrive at the equation
$$1=q_{r+1}q_{r+2}\cdot\cdot\cdot q_s;$$
Which is absurd, since each $q_i >1.$ It follows that $r=s$ and that
$$
p_1=q_1;p_2=q_2,\cdot\cdot\cdot , p_r=q_r;
$$
This makes the two factorizations of n identical.
\end{proof}

\textbf{Note}: It is because of this theorem that we excluded 1 from being a prime. Because if let 1 to be  a prime we wouldn't get a \textbf{unique} factorization.

\begin{lem}[Euclid's lemma]
 Any composite number is divisible by a prime.
\end{lem}
\begin{proof}
For a composite number n, there exists an integer d satisfying the conditions $d\mid{n}$
 and $1<d<n.$ among all such integers d, choose p to be the smallest. Then p must be a prime number. Otherwise, it too would possess a divisor q with $1<q<p;$ but $q\mid{p}$ and $p\mid{n}$ implies that $q\mid{n}$, which contradicts our choice of p as the smallest divisor, not equal to 1, of n. Thus, there exists a prime p with $p\mid{n}.$
\end{proof}

\begin{thm}
If p is a prime and $p\mid{ab},$ then either $p\mid{a}$ or $p\mid{b}$.
\end{thm}

\begin{proof}
If $p\mid{a},$ then we are done, so let us assume that $p\nmid{a}.$ Since the only positive divisors of $p$ ( hence, the only candidates for the value of $gcd(a,p)$ ) are $1$ and $p$ itself, this implies that $gcd(a,p)=1.$ Citing Euclid's lemma, it follows immediately that $p\mid{b}$.
\end{proof}

\begin{thm}
There are an infinite number of primes.
\end{thm}

\begin{proof}
 Write the primes $2, 3, 5, 7, 11\cdot\cdot\cdot $ in ascending order. For any particular prime p, consider the number
                       $ N= (2\cdot3\cdot5\cdot7\cdot11\cdot\cdot\cdot p) +1.$
That is, form the product of all the primes from 2 to p, and increase this product by one. Because $N >1,$ we can use the fundamental theorem to conclude that N is divisible by some prime q. But none of the primes $2, 3, 5,\cdot\cdot\cdot , p$ divides N . For if q were one of these primes, then on combining the relation $ q\mid2\cdot3\cdot5\cdot\cdot\cdot p$ with $q\mid N,$ we would get $ q\mid(N-2\cdot3\cdot5\cdot\cdot\cdot p),$ or what is the same thing, $ q\mid1.$ The only positive divisor of the integer 1 is 1 itself, and since $q >1,$ the contradiction is obvious. Consequently, there exists a new prime $q$ larger than $p$.
\end{proof}
\newpage

\begin{prop}
Some properties that govern prime numbers
\begin{enumerate}
  \item They are all odd with one exception(2).
  \item Their last digit is $1,3,7,9$ with two exceptions(2,5).
  \item They are adjacent of multiple of six with two exceptions(2,3).
\end{enumerate}

\end{prop}
\begin{proof}
It is trivial to show (1) and (2). So let's show (3)
$$ n=6q+r, \qquad ~where~ q\in \mathbb{Z}^+ ~and ~r=\{0,1,2,3,4,5\}$$
If $r=\{0,2,4\}$, then $2|n$ and $n$ can't be prime.\\
If $r=3$, then $3|n$ again $n$ can't be prime.\\
So if n is a prime the remainder $r$ is either $1$ or $5$.
\end{proof}


\begin{thm}[Bertnand’s postulate]
For each natural number $n>1$  there is a prime $p$ such that $n<p<2n$.
\end{thm}
\subsection{Prime Number Theorem}
\begin{thm}

\end{thm}

\bibliographystyle{amsplain}
\begin{thebibliography}{n} %% n is number of items, or the largest label

\bibitem{1}\label{some label - optional} G. H. Hardy and M. Wright
\emph{An Introduction to  the  Theory of Numbers},
6th ed, 2008.

\bibitem{2} Martin Aigner and G\"{u}nter M. Ziegler,
\emph{Proofs from THE BOOK}, Springer, 4th ed, 2009.

\end{thebibliography}

\end{document}
