\documentclass[12pt]{article}

\usepackage[margin=1in]{geometry}  % set the margins to 1in on all sides
\usepackage{graphicx}              % to include figures
\usepackage{amsmath}               % great math stuff
\usepackage{amsfonts}              % for blackboard bold, etc
\usepackage{amsthm}                % better theorem environments
\usepackage{amsthm,amssymb}
\usepackage{xcolor,cancel}
\newcommand\hcancel[2][black]{\setbox0=\hbox{$#2$}%
\rlap{\raisebox{.45\ht0}{\textcolor{#1}{\rule{\wd0}{1pt}}}}#2}

% various theorems, numbered by section

\newtheorem{thm}{Theorem}[section]
\newtheorem{lem}[thm]{Lemma}
\newtheorem{prop}[thm]{Proposition}
\newtheorem{cor}[thm]{Corollary}
\newtheorem{conj}[thm]{Conjecture}

\DeclareMathOperator{\id}{id}

\newcommand{\bd}[1]{\mathbf{#1}}  % for bolding symbols
\newcommand{\RR}{\mathbb{R}}      % for Real numbers
\newcommand{\ZZ}{\mathbb{Z}}      % for Integers
\newcommand{\col}[1]{\left[\begin{matrix} #1 \end{matrix} \right]}
\newcommand{\comb}[2]{\binom{#1^2 + #2^2}{#1+#2}}




\begin{document}

\title{Proofs in Number Theory}
\author{Miliyon T.\thanks{Euclid^1 , Euler^7} \\
Department of Applied Mathematics  \\
Addis Ababa University \\
Ethiopia}
\date{February 24, 2014}
\maketitle

\begin{abstract}
  Number theory is one of the most elegant, abstract and the more beautiful branches of Mathematics.The Greatest mathematician Carl Friedreich Gauss once said that Mathematics is a Queen of Science and Theory of Number is the Queen of Mathematics. Although, Number theory have been considered as non-applicable subject nowadays it is become crucial for Internet Cryptography. The proofs presented here are elementary and beautiful.
\end{abstract}
\section{Basic Results}
\begin{lem}[Bezout's lemma]
For every pair of whole numbers $a$ and $b$ there are two integers $s$ and $t$ such that $as + bt = \gcd(a, b)$.
\end{lem}

\subsection{Euclid's Lemma}
\begin{lem}
\textit{Any composite number is divisible by a prime.}
\begin{proof}
For a composite number $n$, there exists an integer $d$ satisfying the conditions $d\mid{n}$ and $1<d<n.$ among all such integers $d$, choose $p$ to be the smallest. Then $p$ must be a prime number. Otherwise, it too would possess a divisor $q$ with $1<q<p;$ but $q\mid{p}$ and $p\mid{n}$ implies that $q\mid{n}$, which contradicts our choice of $p$ as the smallest divisor, not equal to $1$, of $n$. Thus, there exists a prime $p$ with $p\mid{n}.$
\end{proof}
\end{lem}
\begin{thm}
\textit{If p is a prime and $p\mid{ab},$ then either $p\mid{a}$ or $p\mid{b}$.}

\begin{proof}
If $p\mid{a},$ then we need go no further, so let us assume that $p\nmid{a}.$ Since the only positive divisors of $p$ ( hence, the only candidates for the value of $\gcd (a,p)$) are $1$ and $p$ itself, this implies that $\gcd (a,p)=1.$ Citing Euclid’s lemma, it follows immediately that $p\mid{b}$.
\end{proof}
\end{thm}
\section{Fundamental Theorem of Arithmetic}
\begin{thm}
\textit{Every positive integer $n>1$ is either a prime or can be expressed as a product of primes; this representation is unique, apart from the order in which the factors occur.}
\begin{proof}

Either $n$ is a prime or it is composite. In the first case there is nothing to prove. If $n$ is composite, then there exists a prime divisor of $n$, as we have shown. Thus, $n$ may be written as $n=p_1n_1,$ where $p_1$ is prime and $1<n_1<n.$ If $n_1$ is prime, then we have our representation. In the contrary case, the argument is repeated to produce a second prime number $p_2$ such that $n_1=p_2n_2$; that is,
$$
n=p_1p_2n_2; 1<n_2<n_1:
$$

 If $n_2$ is a prime, then it is not necessary to go further. Otherwise, write $n_2=p_3n_3,$ with $p_3$ a prime; hence,
$$
N=p_1\cdot p_2\cdot p_3\cdot n_3; 1<n_3<n_2:
$$

The decreasing sequence
$n>n_1>n_2>\cdot\cdot\cdot >1$
Cannot continue indefinitely, so that after a finite number of steps $n_k$ is a prime,say $p_k$. This leads to the prime factorization
$n=p_1 p_2∙∙∙ p_k:$
The second part of the proof—the uniqueness of the prime factorization—is more difficult. To this purpose let us suppose that the integer $n$ can be represented as a product of primes in two ways; say,
$n=p_1 p_2\cdot\cdot\cdot p_r=q_1q_2∙∙∙ q_s; r\leq s;$
Where the $p_i$ and $q_j$ are all primes, written in increasing order, so that
$p_1≤ p_2 \leq\cdot\cdot\cdot\leq p_r$ and $q_1\leq q_2 \leq\cdot\cdot\cdot\leq q_s:$
Because $p_1\mid q_1q_2∙∙∙ q_s,$ we know that $p_1|q_k$ for some value of $k$. Being a prime, $q_k$ has only two divisors, $1$ and itself. Because $p_1$ is greater than $1$, we must conclude that $p_1=q_k;$ but then it must be that $p_1\geq q_1.$ An entirely similar argument (starting with $q_1$ rather than $p_1$) yields $q_1\geq p_1,$ so that in fact $p_1=q_1.$ We can cancel this common factor and obtain
$$
p_2 p_3\cdot\cdot\cdot p_r=q_2q_3\cdot\cdot\cdot q_s:
$$
Now repeat the process to get $p_2=q_2;$ cancel again, to see that
$$
p_3 p_4\cdot\cdot\cdot p_r=q_3q_4\cdot\cdot\cdot q_s:
$$
Continue in this fashion. If the inequality $r <s$ held, we should eventually arrive at the equation
$1=q_{r+1}q_{r+2}\cdot\cdot\cdot q_s;$
Which is absurd, since each $q_i >1.$ It follows that $r=s$ and that
$$
p_1=q_1;p_2=q_2,\cdot\cdot\cdot , p_r=q_r;
$$
This makes the two factorizations of $n$ identical.
\end{proof}
\end{thm}

\newpage
\section{Euclid Theorem}

\begin{thm}\textit{There are an infinite number of primes.}
\begin{proof}
Write the primes $2, 3, 5, 7, 11\cdot\cdot\cdot $ in ascending order. For any particular prime $p$, consider the number
                       $ N= (2\cdot3\cdot5\cdot7\cdot11\cdot\cdot\cdot p) +1.$
That is, form the product of all the primes from $2$ to $p$, and increase this product by one. Because $N>1$, we can use the fundamental theorem to conclude that $N$ is divisible by some prime $q$. But none of the primes $2, 3, 5,..., p$ divides $N$ . For if $q$ were one of these primes, then on combining the relation $ q\mid2\cdot3\cdot5\cdot\cdot\cdot p$ with $q\mid n,$ we would get $ q\mid(N-2\cdot3\cdot5\cdot\cdot\cdot p),$ or what is the same thing,$ q\mid1.$ The only positive divisor of the integer $1$ is $1$ itself, and since $q>1$, the contradiction is obvious. Consequently, there exists a new prime $q$ larger than $p$.
\end{proof}
\end{thm}

\section{The $n^{th}$ root of a prime number is irrational.}
\begin{proof}
Suppose not. i.e suppose it is rational, thus we can write $\sqrt[n]{p}=\frac{a}{b}$  where $n \in \mathbb{Z}\geq2$ and  $a,b \in \mathbb{Z}$ and they are relatively prime. Taking a power $n$ both side gives
\begin{align}\label{primerut1}
{p}=\frac{a^n}{b^n}
\end{align}

$$
pb^n=a^n
$$

$$
p\mid a^n  \Rightarrow a\neq 1
$$

From Fundamental theorem of Arithmetic
\begin{equation}
a= \prod_{i=1}^{k} p_i
\end{equation}

$$
a=p_1\cdot p_2\cdot p_3\cdots p_k  ,k\geq 1
$$
$$
\Rightarrow p|(p_1\cdot p_2\cdot p_3\cdots p_k )^n
$$
This implies $p$ divides $p_i$  for some $i$ between $1$ and $k$.\\
Prime number divides prime number
$$ \Rightarrow p=p_i$$
Thus, $ p\mid a$ since $p_i \mid a$
\[
\because p\mid a^n \Rightarrow p|a
\]
Now we can write $a$ as $a=pk$, where $k \in \mathbb{Z}$. Let's substitute this on (\ref{primerut1}).
\[
p=\frac {(pk)^n}{b^n}
\]

$$
pb^n=p^n \cdot k^n
$$

$$
b^n = p^{n-1}\cdot k^n=p\cdot p^{n-2} k^n
$$

$$
b^n=p\cdot p^{n-2} k^n
$$

\noindent Which implies $p\mid b^n$ then by similar argument as the above we can easily show that $p\mid b$.
Now we have shown that $p\mid a$ and $p\mid b$ but this contradict the fact that $a$ and $b$ are relatively prime.\\
Hence our assumption that $\sqrt[n]{p}$ is rational is wrong.\\  $\therefore \sqrt[n]{p}$ is irrational.
\end{proof}


\section{Basel problem}
$$
\zeta(2)=1+ \frac{1}{4}+\frac{1}{9}+\cdot\cdot\cdot =\frac{\pi^2}{6}
$$
\begin{proof}
Consider the function
$$
\frac{\sin ⁡(x)}{x}$$
 which has non zero roots at $\pm\pi,\pm2\pi,\pm3\pi,\pm4\pi, ...$
\\~\\
So we can write this function as infinite product of polynomials like this
$$ \frac{\sin(x)}{x}=(1-\frac{x}{\pi})(1+\frac{x}{\pi})(1-\frac{x}{2\pi})(1+\frac{x}{2\pi})(1-\frac{x}{3\pi})(1+\frac{x}{3\pi})(1-\frac{x}{4\pi})(1+\frac{x}{4\pi})\cdot\cdot\cdot
$$

$$
= (1-\frac{x^2}{\pi^2})(1-\frac{x^2}{4\pi^2})(1-\frac{x^2}{9\pi^2})\cdot\cdot\cdot
$$
Expand this infinite product to get and we are only interested on the coefficient of $x^2$
$$
= 1+(-\frac{x^2}{\pi^2} -\frac{x^2}{(4\pi^2} )-\frac{x^2}{9\pi^2}\cdot\cdot\cdot)+ \cdot\cdot\cdot
$$

$$
= 1-\frac{x^2}{\pi^2}(1+\frac{1}{4}+\frac{1}{9}+\cdot\cdot\cdot)+\cdot\cdot\cdot
$$

\begin{equation} \label{eq:zeta}
\frac{\sin(x)}{x}= 1-\frac{\zeta(2)}{\pi^2}x^2+\cdot\cdot\cdot
\end{equation}
But from Taylor expansion we know that
$$
\sin(x)=x-\frac{x^3}{3!}+\frac{x^5}{5!}-\frac{x^7}{7}!+\cdot\cdot\cdot
$$
Divide both side by $x$ then it becomes
\begin{equation} \label{eq:zeta}
\frac{\sin ⁡(x)}{x}=1-\frac{x^2}{3!}+\frac{x^4}{5!}-\frac{x^6}{7!}+\cdot\cdot\cdot
\end{equation}
Now equate the coefficients of $x^2$ in (2) and (3).
$$
-\frac{\zeta(2)}{\pi^2}=-\frac{1}{3!}
$$

$$
\zeta(2)=\frac{\pi^2}{3!}.
$$
\end{proof}

\section{O(n)=D(n)}

\begin{flushright}
\fbox{%
            \parbox{0.7\linewidth}{%
                    $\textbf{D(n)}$ is the number of ways of writing n as the sum of distinct whole numbers.\\
                    $\textbf{O(n)}$ is the number of ways of writing n as the sum of (not necessarily distinct)odd numbers.
            }%
    }
\end{flushright}
\begin{proof}
 Introduce
$$
P(x)=(1+x)(1+x^2 )(1+x^3 )\cdot\cdot\cdot
$$

$$
=1+x+x^2+(x^3+x^{2+1} )+(x^4+x^{3+1} )+(x^5+x^{4+1}  +x^{3+2} )+ \cdot\cdot\cdot
$$
So
\begin{equation}
p(x)=1+\sum_{n=1}^\infty D(n) x^n
\end{equation}
Introduce
$$
1+a+a^2+a^3+\cdot\cdot\cdot=\frac{1}{(1-a)}
$$
Proof from geometric sum
$$G_n=a_1\frac{(1-r^n)}{(1-r)}$$
But in this case $r=a$ and$a_1=1.$ Therefore
$$
G_n=1\frac{(1-r^n)}{(1-r)}~,
G_n=\frac{1}{(1-a)}-\frac{a^n}{(1-a)}
$$
For $|a|<1$ the second term will be zero.\\
The equation becomes
$$G_n=\frac{1}{(1-a)}$$

Introduce
$$
Q(x)=\frac{1}{(1-x)}\cdot \frac{1}{(1-x^3)}\cdot \frac{1}{(1-x^5)}\cdot\cdot\cdot
$$
\begin{align*}
&=(1+x+x^2+x^3+\cdot\cdot\cdot)(1+x^3+x^6+x^9+\cdot\cdot\cdot)\\
& (1+x^5+x^{10}+x^{15}+\cdot\cdot\cdot)\cdot\cdot\cdot
\end{align*}
\begin{align*}
Q(x)&= (1+x^1+x^{1+1}+x^{1+1+1}+\cdot\cdot\cdot)(1+x^3+x^{3+3}+x^{3+3+3}+\cdot\cdot\cdot)\\
    & (1+x^5+x^{5+5+5}+x^{5+5+5}+\cdot\cdot\cdot)\cdot\cdot\cdot
\end{align*}

So
\begin{equation}
Q(x)=1+\sum_{n=1}^\infty O(n)x^n
\end{equation}

What we have done so far is we introduce two function $P(x)$ and $Q(x)$.Additionally we  have proved that they are actually equal to the following infinite sums.

$$P(x)=(1+x)(1+x^2)(1+x^3)\cdot\cdot\cdot=1+\sum_{n=1}^\infty D(n) x^n$$

$$
Q(x)=\frac{1}{(1-x)}\cdot\frac{1}{(1-x^3)}\cdot \frac{1}{(1-x^5)}\cdot\cdot\cdot=1+\sum_{n=1}^\infty O(n)x^n
$$
Our aim is to show $D(n) =O(n)$. WLOG suppose our generating functions $P(x)$ and $Q(x)$ are equal.
$$P(x) =Q(x)$$

$$1+\sum_{n=1}^\infty D(n) x^n =1+\sum_{n=1}^\infty O(n)x^n$$

$$\Rightarrow D(n) =O(n)$$
Now, we are only expected to show our assumption $P(x)=Q(x)$ is true.

Let's pick $P(x)$ and do some trick
$$P(x)=(1+x)(1)(1+x^2)(1)(1+x^3)\cdot\cdot\cdot$$

$$P(x)=(1+x)(\frac{1-x}{1-x})(1+x^2)(\frac{1-x^2}{1-x^2})(1+x^3)\cdot\cdot\cdot$$

$$ =\frac{\hcancel[red]{(1+x)(1-x)}\hcancel[blue]{(1+x^2)(1-x^2)}(1+x^3)(1-x^3)(1+x^4)(1-x^4)}{~~~~~~~~(1-x)~~~~~~~~~~~\hcancel[red]{(1-x^2)}~~~~~~~~~~~(1-x^3)~~~~~~~~~~~\hcancel[blue]{(1-x^4)}}\cdot\cdot\cdot
$$
If we keep multiplying by this pattern the entire numerator will cancel out and becomes 1. All the expressions with even power will cancel out and the odds left in the de-numerator.
Like this
$$
=\frac{1}{(1-x)}\cdot\frac{1}{(1-x^3)}\cdot \frac{1}{(1-x^5)}\cdot\cdot\cdot
$$
which is $=Q(x)$.\\
Hence we can conclude that
$$D (n) = O (n).$$
\end{proof}

\section{Chinese Remainder Theorem}


The Chinese Remainder Theorem is a result from elementary number
theory about the solution of systems of simultaneous congruences. The
Chinese mathematician Sun-ts\"{\i} wrote about the theorem in the
first century A.D. This theorem has some interesting
consequences in the design of software for parallel processors.


\begin{lem}\label{rings:chinese_remainder_lemma}
Let $m$ and $n$ be positive integers such that $\gcd( m, n) = 1$. Then
for $a, b \in {\mathbb Z}$ the system
\begin{align*}
x & \equiv  a \pmod{m} \\
x & \equiv  b \pmod{n}
\end{align*}
has a solution.  If $x_1$ and $x_2$ are two solutions of the system,
then  $x_1 \equiv x_2 \pmod{mn}$.
\end{lem}


\begin{proof}
The equation $x \equiv a \pmod{m}$ has a solution since $a +km$
satisfies the equation for all $k \in {\mathbb Z}$.  We must show that
there exists an integer $k_1$ such that
\[
a + k_1 m \equiv b \pmod{n}.
\]
This is equivalent to showing that
\[
k_1 m \equiv (b-a) \pmod{n}
\]
has a solution for $k_1$.  Since $m$ and $n$ are relatively prime,
there exist integers $s$ and $t$ such that $ms + nt = 1$.
Consequently,
\[
(b-a) ms = (b-a) -(b-a) nt,
\]
or
\[
[(b-a)s]m \equiv (b-a) \pmod{n}.
\]
Now let $k_1 = (b-a)s$.


To show that any two solutions are congruent modulo $mn$, let $c_1$ and
$c_2$ be two solutions of  the system. That is,
\begin{align*}
c_i & \equiv  a \pmod{m} \\
c_i & \equiv  b \pmod{n}
\end{align*}
for $i = 1, 2$. Then
\begin{align*}
c_2 & \equiv  c_1 \pmod{m} \\
c_2 & \equiv  c_1 \pmod{n}.
\end{align*}
Therefore, both $m$ and $n$ divide $c_1 - c_2$. Consequently,
$c_2 \equiv c_1 \pmod{mn}$.
\mbox{\hspace*{1in}}
\end{proof}



\begin{thm}[Chinese Remainder Theorem]\index{Chinese Remainder Theorem!for integers}
Let $n_1, n_2, \ldots, n_k$ be positive integers such that $\gcd(n_i, n_j)
= 1$ for $i \neq j$. Then for any integers $a_1, \ldots, a_k$, the
system
\begin{align*}
x & \equiv  a_1 \pmod{n_1} \\
x & \equiv  a_2 \pmod{n_2} \\
 &  \vdots  \\
x & \equiv  a_k \pmod{n_k}
\end{align*}
has a solution.  Furthermore, any two solutions of the system are
congruent modulo $n_1 n_2 \cdots n_k$.
\end{thm}

\begin{proof}
We will use mathematical induction on the number of equations in the
system. If there are $k= 2$ equations, then the theorem is true by
Lemma~\ref{rings:chinese_remainder_lemma}. Now suppose that the result is true for a system of $k$
equations or less and that we wish to find a solution of
\begin{align*}
x & \equiv  a_1 \pmod{n_1} \\
x & \equiv  a_2 \pmod{n_2} \\
  & \vdots  \\
x & \equiv  a_{k+1} \pmod{n_{k+1}}.
\end{align*}
Considering the first $k$ equations, there exists a solution that is
unique modulo $n_1 \cdots n_k$, say $a$. Since $n_1 \cdots n_k$ and
$n_{k+1}$ are relatively prime, the system
\begin{align*}
x & \equiv  a \pmod{n_1 \cdots n_k } \\
x & \equiv  a_{k+1} \pmod{n_{k+1}}
\end{align*}
has a solution that is unique modulo $n_1 \ldots n_{k+1}$ by the
lemma.
\end{proof}

\newpage
 \begin{thebibliography}{9}

\bibitem{amsshort}
Euclid's Element:
The Thirteen Book of Euclid translated by Sir Thomas L. Heath
Cambridge University press. 1968.
\bibitem{May}
[Tom Apostle]
An Introduction to Analytic Number Theory
California Institute of Technology. 1976.

\bibitem{notsoshort}
[Jeffry Stopple]
A Primer of Analytic Number Theory From Pythagoras to Riemann
Cambridge  University  Press. 2003.

\bibitem{notsoshort}
[William Dunham]
Euler: The Master of us all.
Mathematical Association of America. 1999.
\end{thebibliography}

\end{document}
