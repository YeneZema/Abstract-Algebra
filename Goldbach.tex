\documentclass[12pt]{article}

\usepackage[margin=1in]{geometry}
\usepackage{amsmath,amsthm,amssymb}
\usepackage{cleveref}

\newcommand{\N}{\mathbb{N}}
\newcommand{\Z}{\mathbb{Z}}

\newenvironment{theorem}[2][Theorem]{\begin{trivlist}
\item[\hskip \labelsep {\bfseries #1}\hskip \labelsep {\bfseries #2.}]}{\end{trivlist}}
\newenvironment{lemma}[2][Lemma]{\begin{trivlist}
\item[\hskip \labelsep {\bfseries #1}\hskip \labelsep {\bfseries #2.}]}{\end{trivlist}}
\newenvironment{exercise}[2][Exercise]{\begin{trivlist}
\item[\hskip \labelsep {\bfseries #1}\hskip \labelsep {\bfseries #2.}]}{\end{trivlist}}
\newenvironment{problem}[2][Problem]{\begin{trivlist}
\item[\hskip \labelsep {\bfseries #1}\hskip \labelsep {\bfseries #2.}]}{\end{trivlist}}
\newenvironment{question}[2][Question]{\begin{trivlist}
\item[\hskip \labelsep {\bfseries #1}\hskip \labelsep {\bfseries #2.}]}{\end{trivlist}}
\newenvironment{corollary}[2][Corollary]{\begin{trivlist}
\item[\hskip \labelsep {\bfseries #1}\hskip \labelsep {\bfseries #2.}]}{\end{trivlist}}

\begin{document}

% --------------------------------------------------------------
%                         Start here
% --------------------------------------------------------------

\title{Goldbach's proof of infinitude of primes}%replace X with the appropriate number

\author{Miliyon T.\\ %replace with your name
(\textit{written in a letter to Euler, July 1730})} %if necessary, replace with your course title

\maketitle
\begin{lemma}{1}\label{lem1}
The Fermat numbers $F_n=2^{2^n}+1$ are pairwise relatively prime.
\end{lemma}

\begin{proof}
It's easy to show by induction that
\begin{align*}
F_{m}-2=F_0 F_1 \cdots F_{m-1}
\end{align*}
This means that if $d$ divides both $F_n \& F_m$ (with $n<m$), then $d$ also divides $F_{m}-2$; so $d$ divides $2$. But every Fermat number is odd. So $d$ is $1$.
\end{proof}
\begin{theorem}{1}
There are infinitely many primes
\end{theorem}

\begin{proof}
Choose a prime divisor $p_n$ of each Fermat number $F_n$.
By (\ref{lem1}) we know these primes are all distinct, showing that there are  infinitely many primes.

\end{proof}

% --------------------------------------------------------------
%     You don't have to mess with anything below this line.
% --------------------------------------------------------------

\end{document}
