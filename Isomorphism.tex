\documentclass[12pt]{article}

\usepackage[margin=1in]{geometry}  % set the margins to 1in on all sides
\usepackage{graphicx}              % to include figures
\pagestyle{empty}
\usepackage[T1]{fontenc}
\usepackage{lmodern}

\usepackage{tikz}
\usetikzlibrary{positioning,shadows,backgrounds}

% various theorems, numbered by section
\usepackage{amsmath,amsfonts,amssymb,amscd,amsthm,xspace}
\newtheorem{thm}{Theorem}[section]
\newtheorem{cor}[thm]{Corollary}
\newtheorem{conj}[thm]{Conjecture}
\theoremstyle{plain}

\newtheorem{example}{Example}[chapter]
\newtheorem{theorem}{Theorem}[chapter]
\newtheorem{lemma}[theorem]{Lemma}
\newtheorem{proposition}[theorem]{Proposition}
\newtheorem{definition}[theorem]{Definition}
\usepackage[centerlast,small,sc]{caption}

\begin{document}


\nocite{}

\title{Isomorphism Theorems}

\author{Miliyon T.}
\date{October 7, 2013}
\maketitle
\section{Group Isomorphism}
\begin{definition}
Two groups $(G, \cdot)$ and $(H, \circ)$ are \textbf{isomorphic}\index{Group!isomorphic} if there exists a bijective map $\phi : G \rightarrow H$ such that the group operation is preserved;  that~is,
\[
\phi( a \cdot b) = \phi( a) \circ \phi( b)\qquad \forall a,b \in G.
\]
If $G$ is isomorphic to $H$, we write $G \cong H$. The map $\phi$ is called an \textbf{isomorphism}.
\end{definition}

\begin{thm}\label{isomorph_theorem_1}
Let $\phi : G \rightarrow H$ be an isomorphism of two groups.  Then the following statements are true.
\begin{enumerate}

\rm \item \it
$\phi^{-1} : H \rightarrow G$ is an isomorphism.

\rm \item \it
$|G| = |H|$.

\rm \item \it
If $G$ is abelian, then $H$ is abelian.

\rm \item \it
If $G$ is cyclic, then $H$ is cyclic.

\rm \item \it
If $G$ has a subgroup of order $n$, then $H$ has a subgroup of order $n$.

\end{enumerate}
\end{thm}

\begin{proof}
Assertions (1) and (2) follow from the fact that $\phi$ is a bijection.  We will prove (3) here and leave the remainder of the theorem to be proved in the exercises.

(3)
Suppose that $h_1$ and $h_2$ are elements of $H$.  Since $\phi$ is onto, there exist elements $g_1, g_2 \in G$ such that $\phi(g_1) = h_1$ and $\phi(g_2) = h_2$.  Therefore,
\[
h_1 h_2 = \phi(g_1) \phi(g_2) =  \phi(g_1 g_2) = \phi(g_2 g_1) = \phi(g_2) \phi(g_1) = h_2 h_1.
\]
\end{proof}


We are now in a position to characterize all cyclic groups.

\begin{thm}\label{isomorph_theorem_2}
All cyclic groups of infinite order are isomorphic to ${\mathbb Z}$.
\end{thm}

\begin{proof}
Let $G$ be a cyclic group with infinite order and suppose that $a$ is a generator of $G$.  Define a map $\phi : {\mathbb Z} \rightarrow  G$ by $\phi : n \mapsto a^n$. Then
\[
\phi( m+n ) = a^{m+n} = a^m a^n = \phi( m ) \phi( n ).
\]
To show that $\phi$ is injective, suppose that $m$ and $n$ are two elements in ${\mathbb Z}$, where $m \neq n$.  We can assume that $m > n$.  We must show that $a^m \neq a^n$. Let us suppose the contrary; that is, $a^m = a^n$. In this case $a^{m - n} = e$, where $m - n > 0$, which contradicts the fact that $a$ has infinite order.  Our map is onto since any element in $G$ can be written as $a^n$ for some integer $n$ and $\phi(n) = a^n$.
\end{proof}

\begin{thm}\label{isomorph_theorem_3}
If $G$ is a cyclic group of order $n$, then $G$ is isomorphic to~${\mathbb Z}_n$.
\end{thm}

\begin{proof}
Let $G$ be a cyclic group of order $n$ generated by $a$ and define a map $\phi : {\mathbb Z}_n \rightarrow  G$ by $\phi(k) = a^k$, where $0 \leq k < n$. The proof that $\phi$ is an isomorphism is one of the end-of-chapter exercises.
\end{proof}

\begin{cor}\label{isomorph_theorem_4}
If $G$ is a  group of order $p$, where $p$ is a prime number, then $G$ is isomorphic to ${\mathbb Z}_p$.
\end{cor}

\begin{proof}
The proof is a direct result of Corollary cosets-theorem7.
\end{proof}

\medskip

The main goal in group theory is to classify all groups; however, it makes sense to consider two groups to be the same if they are isomorphic.  We state this result in the following theorem, whose proof is left as an exercise.

\begin{thm}\label{isomorph_theorem_5}
The isomorphism of groups determines an equivalence relation on the class of all groups.
\end{thm}

Hence, we can modify our goal of classifying all groups to classifying all groups \textbf{up to isomorphism}; that is, we will consider two groups to be the same if they are isomorphic.
\section{Isomorphism Theorem}

Though at first it is not evident that factor groups correspond
exactly to homomorphic images, we can use factor groups to study
homomorphisms. We already know that with every group homomorphism
$\phi: G \rightarrow H$ we can associate a normal subgroup of $G$,
$\ker \phi$; the converse is also true. Every normal subgroup of a
group $G$ gives rise to homomorphism of groups.

Let $H$ be a normal subgroup of $G$. Define the \textbf{natural} or \textbf{canonical homomorphism}
\[
\phi : G \rightarrow G/H
\]
by
\[
\phi(g) = gH.
\]
This is indeed a homomorphism, since
\[
\phi( g_1 g_2 ) = g_1 g_2 H =  g_1 H g_2 H = \phi( g_1) \phi( g_2 ).
\]
The kernel of this homomorphism is $H$.	 The following theorems
describe the relationships among group homomorphisms, normal
subgroups, and factor groups.


\begin{thm}[First Isomorphism Theorem]\label{FirstIsoTheorem}
If $\psi : G \rightarrow H$ is a group homomorphism with $K =\ker
\psi$, then $K$ is normal in $G$. Let $\phi: G \rightarrow G/K$ be
the canonical homomorphism.  Then there exists a unique isomorphism
$\eta: G/K \rightarrow \psi(G)$ such that $\psi =  \eta \phi$.
\end{thm}

 % Proof rewritten for clarification.  Suggested by P. Diethelm.  TWJ 16/8/2013.

\begin{proof}
We already know that $K$ is normal in $G$. Define $\eta: G/K
\rightarrow \psi(G)$ by $\eta(gK) = \psi(g)$.  We first show that
$\eta$ is a well-defined map.  If $g_1 K =g_2 K$, then for some $k \in
K$, $g_1 k=g_2$; consequently,
\[
\eta(g_1 K) = \psi(g_1) = \psi(g_1) \psi(k) = \psi(g_1k) = \psi(g_2)
= \eta(g_2 K).
\]
Thus, $\eta$ does not depend on the choice of coset representatives and the map $\eta: G/K \rightarrow \psi(G)$  is uniquely defined since $\psi =  \eta \phi$.
We must also
show that $\eta$ is a homomorphism, but
\begin{align*}
\eta( g_1K g_2K ) & = \eta(g_1 g_2K) \\
& = \psi(g_1 g_2) \\
& = \psi(g_1) \psi(g_2) \\
& = \eta( g_1K) \eta( g_2K ).
\end{align*}
Clearly, $\eta$ is onto $\psi( G)$.
To show that $\eta$ is one-to-one, suppose that $\eta(g_1 K) =
\eta(g_2 K)$. Then $\psi(g_1) = \psi(g_2)$. This implies that
$\psi( g_1^{-1} g_2 ) = e$, or $g_1^{-1} g_2$ is in the kernel of $\psi$;
hence, $g_1^{-1} g_2K = K$; that is, $g_1K =g_2K$.
\end{proof}


\medskip


Mathematicians often use diagrams called \textbf{commutative diagrams} to describe such theorems. The
following diagram "commutes" since $\psi = \eta \phi$.



\begin{center}


\begin{tikzpicture}[scale=0.8]
\node at (1.5,2) [above] {$\psi$};
\node at (0.25,0.65) {$\phi$};
\node at (2.75,0.65) {$\eta$};
\draw [->] (0,2)  node [left] {$G$} -- (3,2) node [right] {$H$};
\node at (1.5,0) [below] {$G/K$};
\draw [->] (0,1.7) -- (1.3,0);
\draw [->] (1.7,0) -- (3,1.7);
\end{tikzpicture}

\end{center}



\begin{example}
Let $G$ be a cyclic group with generator $g$. Define a map $\phi :
{\mathbb Z} \rightarrow G$ by $n \mapsto g^n$.  This map is a surjective
homomorphism since
\[
\phi( m + n) = g^{m+n} = g^m g^n = \phi(m) \phi(n).
\]
Clearly $\phi$ is onto. If $|g| = m$, then  $g^m = e$. Hence, $\ker
\phi = m {\mathbb Z}$ and ${\mathbb Z} / \ker \phi =  {\mathbb Z} / m {\mathbb Z}
\cong G$. On the other hand, if the order of $g$ is infinite, then
$\ker \phi = 0$ and $\phi$ is an isomorphism of $G$ and ${\mathbb Z}$.
Hence, two cyclic groups are isomorphic exactly when they have the
same order. Up to isomorphism, the only cyclic groups are ${\mathbb Z}$
and ${\mathbb Z}_n$.
\end{example}


\begin{thm}[Second Isomorphism Theorem]
Let  $H$ be a subgroup of a group $G$ (not necessarily normal in $G$)
and $N$ a normal subgroup of $G$.  Then $HN$ is a subgroup of $G$,
$H \cap N$ is a normal subgroup of $H$, and
\[
H / H \cap N \cong HN /N.
\]
\end{thm}


\begin{proof}
We will first show that $HN = \{ hn : h \in H, n \in N \}$ is a
subgroup of $G$.  Suppose that  $h_1 n_1, h_2 n_2 \in HN$. Since
$N$ is normal, $(h_2)^{-1} n_1 h_2 \in N$. So
\[
(h_1 n_1)(h_2 n_2) = h_1 h_2 ( (h_2)^{-1} n_1 h_2 )n_2
\]
is in $HN$. The inverse of $hn \in HN$ is in $HN$ since
\[
( hn )^{-1} = n^{-1 } h^{-1} = h^{-1} (h n^{-1} h^{-1} ).
\]


Next, we prove that $H \cap N$ is normal in $H$. Let $h \in H$ and $n
\in H \cap N$. Then $h^{-1} n h \in H$ since each element is in $H$.
Also, $h^{-1} n h \in N$ since $N$ is normal in $G$; therefore,
$h^{-1} n h \in H \cap N$.


Now define a map $\phi$ from $H$ to $ HN / N$ by $h \mapsto h N$. The
map $\phi$ is onto, since any coset $h n N = h N$ is the image of $h$
in $H$. We also know that $\phi$ is a homomorphism because
\[
\phi( h  h')  = h h' N =  h N h' N =  \phi( h ) \phi( h').
\]
By the First Isomorphism Theorem, the image of $\phi$ is isomorphic to
$H / \ker \phi$; that is,
\[
HN/N = \phi(H) \cong H / \ker \phi.
\]
Since
\[
\ker \phi = \{ h \in H : h \in N \} = H \cap N,
\]
$HN/N = \phi(H) \cong H / H \cap N$.
\end{proof}


\begin{thm}[Correspondence Theorem]\label{CorrespondTheorem}\index{Correspondence Theorem!for groups}
Let $N \vartriangleleft G$. Then $H \mapsto H/N$ is a 1-1 correspondence between the set of subgroups $H$
containing $N$  and the set of subgroups of $G/N$. Furthermore, the normal subgroups of $G$ containing $N$ correspond to normal subgroups of~$G/N$.
\end{thm}



\begin{proof}
Let $H$ be a subgroup of $G$ containing $N$. Since $N$ is normal in
$H$, $H/N$ makes sense.  Let $aN$ and $bN$ be elements of $H/N$. Then
$(aN)( b^{-1} N )= ab^{-1}N \in H/N$; hence, $H/N$ is a subgroup of
$G/N$.

Let $S$ be a subgroup of $G/N$. This subgroup is a set of cosets of
$N$.  If  $H= \{ g \in G : gN \in S \}$, then for $h_1, h_2 \in H$, we
have that $(h_1 N)( h_2 N )= h_1 h_2 N \in S$ and $h_1^{-1} N \in S$.
Therefore, $H$ must be a subgroup of $G$. Clearly, $H$ contains $N$.
Therefore, $S = H / N$. Consequently, the map  $H \mapsto H/N$ is
onto.

Suppose that $H_1$ and $H_2$ are subgroups of $G$ containing $N$ such
that $H_1/N = H_2/N$. If $h_1 \in H_1$, then $h_1 N \in H_1/N$. Hence,
$h_1 N = h_2 N \subset H_2$ for some $h_2$ in $H_2$. However, since
$N$ is contained in $H_2$, we know that $h_1 \in H_2$ or $H_1 \subset
H_2$. Similarly, $H_2 \subset H_1$.  Since $H_1 = H_2$, the map  $H
\mapsto H/N$ is one-to-one.

Suppose that $H$ is normal in $G$ and $N$ is a subgroup of $H$.  Then
it is easy to verify that the map $G/N \rightarrow G/H$ defined by $gN
\mapsto gH$ is  a homomorphism.  The kernel of this homomorphism is
$H/N$, which proves that $H/N$ is normal in $G/N$.


Conversely, suppose that $H/N$ is normal in $G/N$. The homomorphism
given by
\[
G \rightarrow G/N \rightarrow \frac{G/N}{H/N}
\]
has kernel $H$. Hence, $H$ must be normal in $G$.
\end{proof}

\medskip


Notice that in the course of the proof of Correspond Theorem, we have also
proved the following theorem.


\begin{thm}[Third Isomorphism Theorem]\label{ThirdIsoTheorem}\index{Third Isomorphism Theorem!for groups}
Let $G$ be a group and $N$ and $H$ be normal subgroups of $G$ with $N
\subset H$.  Then
\[
G/H \cong \frac{G/N}{H/N}.
\]
\end{thm}


\begin{example}
By the Third Isomorphism Theorem \ref{ThirdIsoTheorem},
\[
{\mathbb Z} / m {\mathbb Z} \cong ({\mathbb Z}/ mn {\mathbb Z})/ (m {\mathbb Z}/ mn
{\mathbb Z}).
\]
Since $| {\mathbb Z} / mn {\mathbb Z} | = mn$ and  $|{\mathbb Z} / m{\mathbb Z}| =
m$, we have $| m {\mathbb Z} / mn {\mathbb Z}| = n$.
\end{example}









\end{document}
