\documentclass[12pt]{article}
\usepackage[margin=1in]{geometry}  % set the margins to 1in on all sides
\usepackage{graphicx}              % to include figures
\usepackage{amsmath}               % great math stuff
\usepackage{amsfonts}              % for blackboard bold, etc
\usepackage{amsthm}                % better theorem environments
\usepackage{amsthm,amssymb}
\usepackage[hidelinks]{hyperref}

\newtheorem{thm}{Theorem}[section]
\newtheorem{lem}[thm]{Lemma}
\newtheorem{prop}[thm]{Proposition}
\newtheorem{cor}[thm]{Corollary}
\newtheorem{conj}[thm]{Conjecture}

\theoremstyle{definition}
\newtheorem{defn}[thm]{Definition}
\newtheorem{defns}[thm]{Definitions}
\newtheorem{con}[thm]{Construction}
\newtheorem{exmp}[thm]{Example}
\newtheorem{exmps}[thm]{Examples}
\newtheorem{notn}[thm]{Notation}
\newtheorem{notns}[thm]{Notations}
\newtheorem{addm}[thm]{Addendum}
\newtheorem{exer}[thm]{Exercise}
\newtheorem{rem}[thm]{Remark}

\DeclareMathOperator{\id}{id}

\newcommand{\bd}[1]{\mathbf{#1}}  % for bolding symbols
\newcommand{\RR}{\mathbb{R}}      % for Real numbers
\newcommand{\ZZ}{\mathbb{Z}}      % for Integers
\newcommand{\col}[1]{\left[\begin{matrix} #1 \end{matrix} \right]}
\newcommand{\comb}[2]{\binom{#1^2 + #2^2}{#1+#2}}
\usepackage{cleveref}	
\begin{document}


\nocite{}

\title{Bijections with composition $(S_A)$}

\author{Miliyon T.}
\date{October 7, 2014}
\maketitle


%\section{$D_4$}

Let $A$ be a non empty set and $S_A$ denotes \textit{the set of all bijections of $A$ onto itself} i.e.
\[
S_A=\{f:A \to A\mbox{ and } f \mbox{ is 1-1 \& onto}\}
\]
Claim $S_A$ with composition, $(S_A,\circ)$, is a group.
\begin{proof}
First of all $S_A\neq \emptyset$ because $\mbox{Id}(x)=x$ for all $x\in A$. Thus, $\mbox{Id}\in S_A$.
\begin{enumerate}
  \item \textbf{Closed}: Let $f,g\in S_A$. Then $f:A \to A$ and $g:A \to A$.
  \[
  f\circ g:A \to A \mbox{ is 1-1 correspondence}
  \]
  Thus, $f\circ g\in S_A$. Hence $S_A$ is closed under $\circ$.
  \item \textbf{Associative}: Let $f,g,h\in S_A$
  \[
  (f\circ g)\circ h=f\circ(g\circ h)\tag{Composition of function is associative}
  \] 
  \item \textbf{Identity}: Let $f\in S_A$. Then $f:A \to A$ is 1-1 and onto. Define 
  \[
  \mbox{Id}:A \to A\quad\mbox{ by } \mbox{Id}(x)=x
  \] 
  then
  \[
  f\circ\mbox{Id}:A \to A
  \]
  Let $x\in A$,
  \begin{align*}
  (f\circ\mbox{Id})(x) & =f(\mbox{Id}(x))=f(x)\Rightarrow f\circ\mbox{Id}=f\\
  (\mbox{Id}\circ f)(x) &=\mbox{Id}(f(x))=f(x)\Rightarrow \mbox{Id}\circ f=f
  \end{align*}
  Hence $f\circ\mbox{Id}=f=\mbox{Id}\circ f$. Thus, the identity element of $S_A$ is $\mbox{Id}:A\to A$ defined by $\mbox{Id}(x)=x$ for all $x$ in $A$.
  \item \textbf{Inverse}: Let $f\in S_A$. Then $f:A \to A$ is 1-1 and onto. Then $f^{-1}:A \to A$ is 1-1 and onto. This implies $f^{-1}\in S_A$.  
\end{enumerate}
Therefore $(S_A,\circ)$ is a group.
\end{proof}

\begin{rem}
If $A=\{1,2,3,\ldots,n\}$, then instead of $S_A$ we write $S_n$.
\end{rem}

\end{document}
